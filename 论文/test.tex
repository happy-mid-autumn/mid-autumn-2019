\documentclass[UTF8]{ctexart}
\usepackage{graphicx}

\title{机场的出租车问题研究}
\author{杨哲,于佳睿,吴健宗}

\begin{document}
	\maketitle
	\begin{abstract}
		本文对机场的出租车排队、载客等相关问题进行研究。
		
		针对问题一、二,我们首先分析了影响出租车司机的决策因素,
		
		针对问题三,我们对于现行两车道的情况提出了两种方案: 方案A: 两并行车道都设置上车口,不允许后车超越堵塞它的前车; 方案B: 仅一侧设置上车口,允许后车使用另一条车道超越堵塞它的前车。我们使用排队论的数学模型,并使用我们在机场实际测得的数据对模型效果进行评估,使用软件拟合,找出最优的上车口数量设置,并比较两个方案的优劣。
		
		针对问题四
	\end{abstract}
	
	\section{问题背景与重述}
	随着人们生活水平的日益提高,对于机场下机的旅客,出租车日益成为主要的交通工具之一。送客到机场的出租车司机都将会面临两个选择:
	
	(A) 前往到达区排队等待载客返回市区。
	
	(B) 直接放空返回市区拉客。 
	
	A情况下司机只能在蓄车池里面排队,可能损失潜在的载客收益,B情况下出租车司机会付出空载费用。司机可观测到的确定信息只有在某时间段抵达的航班数量和“蓄车池”里已有的车辆。通常司机的决策与其个人的经验判断有关,比如在某个季节与某时间段抵达航班的多少和可能乘客数量的多寡等,这样的评估多是定性的,而非定量的。在很多数据可以获取的今天,利用数据帮助出租车司机完成尽可能准确的决策成为可能。
	
	如果乘客在下飞机后想“打车”,就要到指定的“乘车区”排队,按先后顺序乘车。对于机场管理人员来说,影响调度情况的最主要因素就是乘车区上车点的数目和上车点的设置方式,通过合理设置数目和乘车口的位置可以显著加快服务进度。
	
	在上述背景下题目要求完成以下任务
	
	\textbf{任务一:} 分析研究与出租车司机决策相关因素的影响机理,建立出租车司机选择决策模型,并给出司机的选择策略。
	
	\textbf{任务二:} 收集国内某一机场及其所在城市出租车的相关数据,给出该机场出租车司机的选择方案,并分析模型的合理性和对相关因素的依赖性。
	
	\textbf{任务三:} 某机场“乘车区”现有两条并行车道,建立模型,指出管理部门应如何设置“上车点”,并合理安排出租车和乘客,在保证车辆和乘客安全的条件下,使得总的乘车效率最高。
	
	\textbf{任务四:} 机场的出租车载客收益与载客的行驶里程有关,允许出租车多次往返载客。短途载客的旅客制定一个可行的“优先” 安排方案使得短途载客的出租车具有一定的“优先权”,使得这些出租车的收益尽量均衡。
	\section{任务一、二: 基于计算机模拟和Logistic\\回归的出租车决策模型及依赖因素分析}
	\subsection{模型假设}
	\subsection{符号约定}
	列一张表
	\subsection{问题分析、数据采集与数据预处理}
	注意配图说明
	\subsection{模型建立与分析}
	给出详细的定义和数学公式
	\subsection{模型的评价与改进}
	
	\section{任务三: 基于排队论的旅客平均等待时间模型}
	\subsection{模型假设}
	\textbf{1.} 假定为了方便管理出租车的上车点集中分布在同一路段。
	
	\textbf{2.} 假定出于安全考虑在出租车上车点路段限速,出租车跟车行驶一个车位所需的时间是一个常数。
	
	\textbf{3. }为了考察队列模型对峰值的承载能力且为了研究方便,假定旅客的到达率始终为某一峰值到达率。
	
    \textbf{4.} 假定旅客出行乘出租车每组只能是1,2,3,4人,将旅客按组分割,并假定每种旅客组的乘车服务时间服从各自的正态分布,旅客的到达服从泊松分布。
    
    \textbf{5. }服务率总是高于到达率,队列不会达到无穷长。
	\subsection{符号约定}
		\begin{tabular}{ccc}
			\hline
			符号 & 意义 \\
			\hline
			$k$ & 道路一侧的上车口数目 \\
			$t_{in}$ & 出租车驶入乘车段后驶过一个车位的时间 \\ 
			$t_{car}$ & 在没有阻塞的情况下一辆车的净服务时间 \\
			$t_{car_j}$ & 在没有阻塞的情况下第j列一辆车的净服务时间 \\
			$t_{car_j(real)}$ & 在有阻塞的情况下第j列一辆车的净服务时间 \\
			$f_{*}(t)$ & *物理量的密度函数 \\
			$F_{*}(t)$ & *物理量的分布函数 \\
			$t_{total_j}$ & 第j列总的服务时间 \\
			$\sigma_i$ & 第i类分组旅客的正态分布方差 \\
			$\mu_i$ & 第i类分组旅客的正态分布均值 \\
			$P_i$ &  第i类分组旅客的出现概率 \\
			$\lambda$ & 旅客组的总到达率 \\
			$\lambda_j$ & 第j列旅客组的总到达率 \\
			$L_s$ &  队长 \\
			$L_{s_j}$ & 第j列队长 \\
			$W_s$ &  等待时间 \\
			$W_{s_j}$ & 第j列队长 \\
			\hline	
			\label{tab:test1}
		\end{tabular}

		注: $j \in {N}, 0<j \leq k; i \in \{1,2,3,4,5\}$
	
	\subsection{问题分析、数据采集与数据预处理}
	\textbf{1. 问题分析}
	
	本题指出是在出现排队的情况,上面两个问题已经给出了一个出租车是否进入蓄车池的决策,本题指出存在人等车和车 人的两种情况,任务一、二已经给出了车的决策方式,那么本题着重考虑的是如何调度乘客排队,和车辆接客的方式来使得每个候车乘客的平均等待时间最小。一个直觉的推断就是开的上车口越多,服务速率越快,等待时间越短。但是由于分批放入接客的管理模式,实际上考虑出租车驶入场内需要的时间会随上车口的增多而增大,而且考虑管理成本的问题,k并不是越大越好。我们就是要找到尽可能小的k值使之可以满足需要,管理成本最低,且要考虑旅客到达率动态调整。另外旅客是按照1,2,3,4的分组到达乘出租车所以我们按照"组"为单位考虑。
	
	\textbf{2. 数据采集与数据预处理}
	1. 根据机场的实际情况,发现上车口仅有2-4名管理人员,所以$k$值不会过大,否者会出现管理混乱,秩序不良的情况。
	
	\begin{table}
		\centering
		\begin{tabular}{cccccccc}
			\hline
			航空公司 & 国航 & 东航 & 南航 & 春秋 & 吉祥 & 深航 &  海航 \\
			\hline
			上座率 & 80.60 &  82.29    &   82.44   & 89.01&  86.24 & 84.79   & 82.69 \\
			\hline
		\end{tabular}
		\caption{\cite(航空公司上座率数据集)}
		
	\end{table}
	
	\begin{table}
		\centering
		\begin{tabular}{cccccccc}
			\hline
			机型 & B737 & A320 & A321 & A330 & A319 & ERJ &  其他 \\
			\hline
			比例 & 41\% &  26\%   &   10\%   & 6\% & 6\% & 4\% & 7\% \\
			\hline
			核载 & 189 & 150 & 185 & 275.6 & 124 & 50 & -- \\
			\hline
		\end{tabular}
		\caption{各机型的核载数据}
		
	\end{table}
	 2. 根据表1表2,估计飞机的旅客上座率和各机型核载,计算加权平均载客量173.1人。
	
	3. 我们在合肥新桥机场实地测量了15:30-16:30的实时数据(见附录),估测乘出租车离开的旅客组到达率.
	\subsection{模型建立与分析}
	由于在一天之中旅客的到达率$\lambda$的变化很大,我们要研究的是上车点的数目和设置方式,所以将平均等待时间表示为$k$和$\lambda$的函数,即平均等待时间是$f(k, \lambda)$,我们的目的是在给定不同的$\lambda$的情况下,研究方案A和B中$k$分别取什么值时,使得$f$最小,即待求(优化)的目标函数$K(\lambda) =  {arg}_k \max f(k, \lambda)$
	\subsection{模型的评价与改进}
	
	\section{任务四: 基于梯度下降法的短途车优先策略}
	\subsection{问题重述}
	出租车载客收益与载客的行驶里程有关,行驶里程越长,收益越高。花费数小时排队的出租车如果只接到了短程的旅客(以下简称短程车),相较于其他出租车来说,收益明显低于平均值,为了平衡出租车之间的收益差距,机场应当鼓励一部分短程车在完成短程载客后回到机场,给予他们一定的“优先权”,即允许跳过排队过程,优先接待候车区的旅客。
	
	\subsection{问题目标}
	设计算法,通过给予返回机场的短程车“优先权”的方法,平衡短程车与正常排队出租车的收益,达到以下效果:\\
	1. 使双方的收益尽量一致。\\
	2. 短途车的“优先权”会使得队列整体行进速度变慢,因此需要控制拥有“优先权”的短程车的数量\\
	
	\subsection{模型假设}
	为了简化模型,现根据实际情况做出如下假设:\\
	\textbf{1.} 在载客距离较长的情况下,可以忽略起步价的影响。\\
	\textbf{2.} 出租车的行驶速度一定且个体之间没有差异,即载客距离,行驶时间,收益,三者成正比。\\
	\textbf{3.} 从机场搭乘出租车的乘客的乘车路程呈现以机场到市中心的距离为均值的正态分布。\\
	\textbf{4.} 驾驶员与乘客均按照规定行事,不存在违规行为。\\
	\textbf{5.} 短程车载客时的收益速率等于市区载客的收益速率(以下称正常收益速率)\\
	注: 收益速率:单位时间内的收益
	\subsubsection{假设合理性解释}
	以合肥新桥国际机场为例(其他城市的机场作上述假设也是合理的)\\
	\textbf{1.} 合肥市小型出租车的起步价为8元/2.5公里,公里租价为1.3元,而从机场出发的出租车,行程普遍大于等于10公里(即使是短程车),因此我们可以近似认为起步价的影响可以忽略,即在行驶过程中,载客距离与收益成正比。\\
	\textbf{2.} 目前城市的各个主干道都有限速规范,不同出租车个体之间的速度差距不会过大。第二,出于实践考虑,我们希望用时间代替路程,这样仅仅通过计时,就可以较为容易地判断一辆出租车是不是短程车,而不用进行复杂的定位处理。第三,在建立模型时,用时间来衡量收益可以有效地简化模型。\\
	\textbf{3.} 以合肥市人口分布图为例\\
	![](合肥人口分布图.jpg)
	可以看到,合肥市的人口分布以市中心的庐阳区,瑶海区,包河区等地区为核心,逐渐向四周扩散,人口密度逐渐减小,基本符合正态分布。\\
	\textbf{4.} 为了简化模型,不考虑违规情况 \\
	\textbf{5.} 短程载客与市区内载客的路程长短相当,可以看做收益相当 \\
		
	\subsection{符号约定}
	\begin{tabular}{ccc}
		\hline
		符号 & 意义 \\
		\hline
	    $t_p$ & 机场出租车的平均排队时间 \\
	    $t_s$ & 短程车的平均载客时间(也等于从目的地回到机场的时间) \\
	    $t_{smin}$ & 出租车被界定为短程车的最短载客时间 \\
	    $t_{smax}$ & 出租车被界定为短程车的最长载客时间 \\
	    $p$ & 短程车占所有出租车的比重(概率) \\
		$p_{pre}$ & 预计的短程车占比 \\
		 $f(t)$ & 出租车载客时间分布的概率函数 \\
		$t_\mu$ & 长程车平均载客时间 \\
		$\lambda_i (i = 1,2)$  & 损失函数的约束参数 \\
		\hline
	\end{tabular}
	\subsection{模型建立}
	\subsubsection{建立收益均衡方程}
	从一辆短程车的视角出发,从它开始在机场排队开始,到回到机场为止,整个流程为:\\
		1. 排队:平均用时$t_p$ \\
		2. 载短程客:平均用时$t_s$ \\
		3. 从下客点回到机场:平均用时$t_s$
		
	而一辆非短程车(长程车)的流程是:\\
		1. 排队:平均用时$t_p$ \\
		2. 载长程客(一般是去市区):平均用时$t_\mu$ \\
		3. 在市区继续载客,恢复正常收益速率 \\
	
	于是,短程车总用时$t_p + 2t_s$,其中$t_s$时间处在正常收益速率,另外的$t_p + t_s$处于无收益状态。短程车回到机场后,如果机场给予短程车优先权,那么它可以直接载客回到市区,恢复正常收益速率。
	
	相比之下,长程车排队的$t_p$时间内无收益,载客到达市区之后恢复正常收益速率。
	
	于是,短程车的无收益时间大于长程车的无收益时间,为了平衡二者的差距,给予短程车“优先权”,允许短程车无需排队,直接在上车点载客。
	
	短程车的“优先载客”在长程车看来是一种“插队行为”,会导致整个队列的行进速度变慢,排队时间变长。
	
	若短程车占所有出租车的比重为$p$,则长程车的排队时间变为了$$t_q(1 + p) $$
	
	短程车的空载时间一共为$$t_s + t_q $$
	
	为了使长短程车收益均衡,则两式相等$$t_q(1 + p) = t_s + t_q$$
	
	即$$t_s = p*t_q$$
	
	\subsubsection{建立计算模型}
	$t_q$为已知参量(由统计数据得出),$t_s$和$p$都依赖于对短程车的界定,即需要给出$t_{smin}$和$t_{smax}$
	
	$$p = \int_{t_{smin}}^{t_{smax}}f(t)dt$$
	$$t_s = p*t_q = \int_{t_{smin}}^{t_{smax}}f(t)dt*t_q$$
	
	在实际操作上,我们真正关注的是$t_{smin}$和$t_{smax}$,得到了这两个量,我们就可以界定短程车的范围,从而实施“优先权”策略。
	
	为了选取合适的$t_{smin}$和$t_{smax}$,需要一个量化的方法来判断所选择的$t_{smin}$和$t_{smax}$的好坏。为此,我们选择使用损失函数$J$来衡量短程车界定域的好坏,对每一对{$t_{smin},t_{smax}$},有
	
	$$J_1 = (\frac{t_{smin}+t_{smax}}{2} - t_s)^2$$
	$$J_2 = \lambda_1 t_{smax}^2$$
	$$J_3 = \lambda_2 (p - p_{pre})^2$$
	$$J = \sum_{i=1}^{3}J_i$$
	
	$J1$限定了能够使长短程车收益平衡的$t_s$需要在短程车界定区域内部,如果不作限制,界定出来的短程车可能会获得过多的收益或者过少的收益。
	
	$J2$限制了$t_{smax}$不能过大,否则界定出来的短程车实际上载客距离和长程车一致(去市区),然后又回到了机场,不符合常理。
	
	$J3$首先给出了一个预计的短程车占比$p_{pre}$,我们希望返回机场的短程车不要过多,否则短程车的“插队”行为会使队列的行进速度变得十分缓慢,因此预先设定一个短程车占比值,并且希望界定出来的短程车占比不要超过$p_{pre}$太多。 
	
	
	\subsection{模型求解}
	4.3节给出了需要进行优化的损失函数$J$,本节中采用梯度下降法对该损失函数进行优化。
	
	\subsubsection{梯度下降法}
	梯度下降法是机器学习领域的常用的寻找最优参数的方法,核心思想为:从一个初始点开始,每次朝着损失函数下降最快的的方向移动一次待优化的参数(在本文中即为$t_{smin}$和$t_{smax}$),反复此过程,最终待优化参数会停在损失函数的极小值点上。
	
	注:极小值点可能仅仅是局部最小值,但是可以通过改变学习率与初始值来寻找全局最小值
	
	伪代码为:
	for i = 1:iter_num
	
	$$ t_{smin} = t_{smin} - \alpha*\frac{\partial{J}}{\partial{t_{smin}}}$$
	
	$$ t_{smax} = t_{smax} - \alpha*\frac{\partial{J}}{\partial{t_{smax}}}$$
	
	end
	
	注:其中$\alpha$为学习率,即梯度下降的速率
	
	\subsubsection{参数赋值}
	为了求解该模型,需要对一些参数进行赋值,结合实际情况,对一些参数赋值如下表
	
	\begin{tabular}{ccc}
		\hline
		参数 & 值 & 单位\\
		\hline
		$t_q$ & 180 & s \\
		$p_{pre}$ & 0.01:0.01:0.25 & 无\\
		$f(t)$ & 服从均值为45,标准差为10的正态分布 & 无\\
		\hline
	\end{tabular}
	
	![](合肥市行驶时间分布.jpg)
	
	注: 0.01:0.01:0.25表示从0.01到0.25,每隔0.01取一个值
	
	\subsection{求解结果}
	运行梯度下降程序(附件:梯度下降代码),迭代次数为200次,得到的学习曲线为
	
	![](Jp=0.25.jpg)
	
	$$p_{pre}=0.25$$
	
	![](Jp=0.12.jpg)
	
	$$p_{pre}=0.12$$
	
	注:纵轴的$J_{history}$为损失函数$J$
	
	梯度下降在100次迭代附近就已经达到了最大效率,之后的迭代下损失函数几乎不变。
	
	保持200次的迭代次数不变,调整$p_{pre}$从0.01到0.25,最终得到的损失函数$J$与$p_{pre}$关系如下图
	
	![](J-P.jpg)
	
	$J$的三个组成部分中,$J1$和$J3$都随$p_{pre}$的增加先减少后增加,而$J2$近乎不变,$J$的变化规律和$J1,J3$相似。
	
	$J$的终值在$p_{pre}=0.12$时达到最小值
	
	$p_{pre}=0.12$时,
	$$
	J1 = 0.012, \\
	J2 = 10.993, \\
	J3 = 0.043, \\
	J = 11.048, \\
	t_{smin} = 9.512, \\
	t_{smax} = 33.155,\\
	t_s = 21.226,\\
	p = 0.117 \\
	$$
	
	注:所有结果保留小数点后三位
	
	

	\subsection{结论}
	由求解结果得出,在$p_{pre}=0.12$时,短程车优先的策略所得到的效果最好,并且可以规定载客时间在10-33分钟之间的出租车为短程车,机场给予其优先权,让其无需排队,直接再次载客。
	
	按照这个策略,短程车占所有出租车的比重约为0.117,上海浦东国际机场的统计为0.15(**引用**),减少了约0.3,队列可以行进地更快。
	
	\subsection{模型的不足与改进}
	\subsubsection{模型不足}
	1. 设定了一些简化模型的假设,虽然简化了模型建设与求解,但是一定程度上也减少了模型与实际的切合性。

	2. 缺少足够的真实数据,所得结果与实际情况有所偏差,但是该模型可以直接带入真实数据求解。
	
	3. 人为设定$J1,J2,J3$,可能有更多的考虑因素没有涉及。
	
	4. 虽然只有0.12占比的短途车,但是在实际情况中,由于出租车基数庞大,会一直有短途车返回机场,并且获得“优先权”而持续占据上车点,导致正常排队的出租车很难上客。
	
	\subsubsection{模型改进}
	针对短途车始终具有“优先权”,导致正常排队的出租车无法上车的情况,本文采用“短途-长途轮转”的方式预防“**饥饿问题**”。
	
	注:饥饿问题:拥有优先权的少量个体会堵塞大多数个体
	
	“短途-长途轮转”操作方式:给回到机场的短途车单独设置一个队列,在上车点处,管理员调车时,一次调短途车(如果有),一次调正常排队的车,两条队列轮流调度,可以保证正常队列不会被一直堵塞。
	
	\subsection{数据集}
	
	15.30 - 16.30 的到客数据(前50)
	 
	16.30 - 15.30 旅客打车时间数据(后50)
	
	汽车滑动一个身位时间

	\begin{flushleft}
		\begin{thebibliography}{123456}  
			\addtolength{\itemsep}{-1.5ex}
			\bibitem {JW1}A. Ahmed, K. Yu, W. Xu, Y. Gong, and E. Xing. Training hierarchical feed-forward visual recognition models using transfer learning from pseudo-tasks. In ECCV (3), pages 69–82, 2008.
			\bibitem {JW2}M.-R. Amini and P. Gallinari. Semi supervised logistic regression. In ECAI, pages 390–394, 2002.
			\bibitem {JW3} L. Bo, X. Ren, and D. Fox. Unsupervised Feature Learning for RGB-D Based Object Recognition. In ISER, June 2012.
			\bibitem {JW4}	上座率:来自 http://www.lvjie.com.cn/company/2019/0429/11846.html 和几家航空公司的年报
			\bibitem {JM5} 	核载 来自 https://data.variflight.com/reports/Fileservice/fileDownload?fileId=76
			
			
		\end{thebibliography} 
		
	\end{flushleft}
	
	
\end{document}